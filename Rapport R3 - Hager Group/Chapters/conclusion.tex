Le projet initié par le groupe Hager offre une opportunité enrichissante pour explorer les applications pratiques de l'intelligence artificielle au sein de notre formation. En traitant un problème concret tel que l'enrichissement de leur base de données pour prévenir les défauts d'arc, ce projet représente une étape cruciale dans notre parcours d'apprentissage. Notre rapport contribue significativement à résoudre les défis réels auxquels font face des industries comme Hager. Nous avons pu élaborer nos premiers modèles et obtenir quelques résultats. Nous pouvons à présent utiliser la puissance de calcul proposée par notre école Télécom Physique Strabourg grâce à l'InnovLab pour effectuer des phases plus importante d'apprentissage. Nous allons aussi pouvoir nous orientier sur conseil de notre tuteur entreprise vers les Transformers. 