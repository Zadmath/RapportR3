Dans le cadre de notre collaboration novatrice avec Hager Group, ce projet d'ingénierie élargit les limites de l'efficacité de l'AFDD (Arc Fault Detection Device) en utilisant les capacités innovantes des réseaux génératifs antagonistes (GAN). La mission que nous avons reçue de Hager Group, un leader mondial dans les solutions électriques, consiste à explorer l'intégration de données synthétiques générées par des GAN afin de renforcer la détection des arcs électriques dans leurs systèmes.

Les GAN, des experts dans le domaine de l'apprentissage profond, permettent de générer des données synthétiques de bonne qualité. Nous visons principalement à exploiter ces progrès technologiques afin d'enrichir les ensembles de données déjà existantes de Hager Group. De cette manière, notre objectif est d'accroître la capacité des systèmes d'AFDD à détecter et à diagnostiquer de manière précise les anomalies dans les systèmes électriques, ce qui contribue à la fiabilité et à la sécurité des stations électriques.